%-------------------------
% Resume in Latex
% Author : Charles-Olivier Ipperciel
%------------------------

\documentclass[letterpaper,11pt]{article}

\usepackage{latexsym}
\usepackage[empty]{fullpage}
\usepackage{titlesec}
\usepackage{marvosym}
\usepackage[usenames,dvipsnames]{color}
\usepackage{verbatim}
\usepackage{enumitem}
\usepackage[hidelinks]{hyperref}
\usepackage{fancyhdr}
\usepackage[english]{babel}
\usepackage{tabularx}
\input{glyphtounicode}


%----------FONT OPTIONS----------
% sans-serif
% \usepackage[sfdefault]{FiraSans}
% \usepackage[sfdefault]{roboto}
% \usepackage[sfdefault]{noto-sans}
% \usepackage[default]{sourcesanspro}

% serif
% \usepackage{CormorantGaramond}
% \usepackage{charter}


\pagestyle{fancy}
\fancyhf{} % clear all header and footer fields
\fancyfoot{}
\renewcommand{\headrulewidth}{0pt}
\renewcommand{\footrulewidth}{0pt}

% Adjust margins
\addtolength{\oddsidemargin}{-0.5in}
\addtolength{\evensidemargin}{-0.5in}
\addtolength{\textwidth}{1in}
\addtolength{\topmargin}{-.5in}
\addtolength{\textheight}{1.0in}

\urlstyle{same}

\raggedbottom
\raggedright
\setlength{\tabcolsep}{0in}

% Sections formatting
\titleformat{\section}{
  \vspace{-4pt}\scshape\raggedright\large
}{}{0em}{}[\color{black}\titlerule \vspace{-5pt}]

% Ensure that generate pdf is machine readable/ATS parsable
\pdfgentounicode=1

%-------------------------
% Custom commands
\newcommand{\resumeItem}[1]{
  \item\small{
    {#1 \vspace{-2pt}}
  }
}

\newcommand{\resumeSubheading}[4]{
  \vspace{-2pt}\item
    \begin{tabular*}{0.97\textwidth}[t]{l@{\extracolsep{\fill}}r}
      \textbf{#1} & #2 \\
      \textit{\small#3} & \textit{\small #4} \\
    \end{tabular*}\vspace{-7pt}
}

\newcommand{\resumeSubSubheading}[2]{
    \item
    \begin{tabular*}{0.97\textwidth}{l@{\extracolsep{\fill}}r}
      \textit{\small#1} & \textit{\small #2} \\
    \end{tabular*}\vspace{-7pt}
}

\newcommand{\resumeProjectHeading}[2]{
    \item
    \begin{tabular*}{0.97\textwidth}{l@{\extracolsep{\fill}}r}
      \small#1 & #2 \\
    \end{tabular*}\vspace{-7pt}
}

\newcommand{\resumeSubItem}[1]{\resumeItem{#1}\vspace{-4pt}}

\renewcommand\labelitemii{$\vcenter{\hbox{\tiny$\bullet$}}$}

\newcommand{\resumeSubHeadingListStart}{\begin{itemize}[leftmargin=0.15in, label={}]}
\newcommand{\resumeSubHeadingListEnd}{\end{itemize}}
\newcommand{\resumeItemListStart}{\begin{itemize}}
\newcommand{\resumeItemListEnd}{\end{itemize}\vspace{-5pt}}

%-------------------------------------------
%%%%%%  RESUME STARTS HERE  %%%%%%%%%%%%%%%%%%%%%%%%%%%%


\begin{document}

%----------HEADING----------
% \begin{tabular*}{\textwidth}{l@{\extracolsep{\fill}}r}
%   \textbf{\href{http://sourabhbajaj.com/}{\Large Sourabh Bajaj}} & Email : \href{mailto:sourabh@sourabhbajaj.com}{sourabh@sourabhbajaj.com}\\
%   \href{http://sourabhbajaj.com/}{http://www.sourabhbajaj.com} & Mobile : +1-123-456-7890 \\
% \end{tabular*}

\begin{center}
  \textbf{\Huge \scshape Charles-Olivier Ipperciel} \\ \vspace{1pt}
  \small 438-827-2307 $|$ \href{mailto:charlesolivieripperciel@gmail.com}{\underline{charlesolivieripperciel@gmail.com}} $|$ 
  \href{https://www.linkedin.com/in/coipp/}{\underline{linkedin.com/in/coipp/}} $|$
  \href{https://charlesoipperciel.github.io/PortfolioReact/}{\underline{charlesoipperciel.github.io/PortfolioReact}}
\end{center}


%-----------EDUCATION-----------
\section{Éducation}
  \resumeSubHeadingListStart
  \resumeSubheading
      {Université de Sherbrooke}{Québec, Canada}
      {Maîtrise en informatique, temps partiel}{Sep 2025 -- Août 2027}
    \resumeSubheading
      {Université de Sherbrooke}{Québec, Canada}
      {Baccalauréat en informatique, programme COOP}{Jan 2022 -- Avril 2025}
    \resumeSubheading
      {HEC Montréal}{Québec, Canada}
      {Baccalauréat en administration des affaires, majeure de 2 ans}{Sep 2019 -- Déc 2021}
  \resumeSubHeadingListEnd


%-----------EXPERIENCE-----------
\section{Expérience}
  \resumeSubHeadingListStart

    \resumeSubheading
      {Développeur Full Stack}{Mai 2024 -- Présent}
      {MEDomicsLab}{Hybride}
      \resumeItemListStart
        \resumeItem{Développement de MEDomicsLab, une plateforme open-source d'IA en santé.}
        \resumeItem{Optimisation des procédés et intégration de MongoDB pour gérer des ensembles de données plus importants.}
        \resumeItem{Développement sur le frontend (React) et le backend (Python) pour améliorer les appels à la base de données.}
        \resumeItem{Stage à temps plein qui s'est transformé en un poste à temps partiel.}
        \resumeItem{Technologies utilisées : Python, MongoDB, React.js, Git.}
      \resumeItemListEnd

    \resumeSubheading
      {Développeur Backend}{Août 2023 -- Déc 2023}
      {Sherweb}{À distance}
      \resumeItemListStart
        \resumeItem{Développement au sein de l'équipe de facturation, correction de bugs et surveillance quotidienne du logiciel.}
        \resumeItem{Intégration du logiciel Braintree pour le traitement des CC, remplaçant Paysafe dans une base de code en C\#.}
        \resumeItem{Travail dans un environnement Agile avec des réunions quotidiennes.}
        \resumeItem{Technologies utilisées : C\#, .NET, Azure DevOps, Datadog, Octopus, Git, MySQL.}
      \resumeItemListEnd

    \resumeSubheading
      {Développeur Full Stack}{Jan 2023 -- Mai 2023}
      {Qwatro Inc.}{Sur site}
      \resumeItemListStart
        \resumeItem{Création d'une application pour le suivi des heures de travail des employés sur les chantiers de construction.}
        \resumeItem{Développement de l'interface utilisateur et du backend en Python avec MongoDB comme base de données.}
        \resumeItem{Assistance dans le choix d'un ERP pour la transformation numérique.}
        \resumeItem{Technologies utilisées : Python, MongoDB, Git.}
      \resumeItemListEnd

  \resumeSubHeadingListEnd

%-----------PROJECTS-----------
\section{Projets}
    \resumeSubHeadingListStart
      \resumeProjectHeading
          {\textbf{Vitopia – Simulations Visuelles d'Entités Informatiques et Biologiques} $|$ \emph{C++, OpenGL, Git}}{En cours}
          \resumeItemListStart
            \resumeItem{Développement d'une simulation 2D observant l'évolution d'organismes dans des sociétés complexes.}
            \resumeItem{IA, modélisation des comportements écologiques, des réactions aux stimuli et des interactions économiques.}
            \resumeItem{Le projet sera déployé en ligne et disponible sur GitHub.}
          \resumeItemListEnd

      \resumeProjectHeading
          {\textbf{Coded Kingdom} $|$ \emph{Godot, Python, Aseprite, Git}}{Projet de 130 heures}
          \resumeItemListStart
            \resumeItem{Conception d'un jeu en pixel art qui enseigne la programmation à travers un gameplay interactif.}
            \resumeItem{Les joueurs doivent écrire du code et l'exécuter pour faire progresser leurs personnages dans les niveaux.}
            \resumeItem{C'était l'un de mes projets de fin de baccalauréat, avec tout le code et la dernière version disponibles sur GitHub.}
          \resumeItemListEnd

          \resumeProjectHeading
          {\textbf{Calculateur de Dosage d'Insuline} $|$ \emph{React, Firebase, HTML/CSS, Git}}{Projet Personnel}
          \resumeItemListStart
            \resumeItem{Développement d'une application pour les patients récemment diagnostiqués avec un diabète de type 1.}
            \resumeItem{L'application permet de calculer les doses d'insuline nécessaires et est utilisée par des résidents pédiatriques pour vérifier leurs résultats de calcul à l'Hôpital Fleurimont, à Sherbrooke.}
            \resumeItem{Hébergée et déployée sur Firebase.}
          \resumeItemListEnd
    \resumeSubHeadingListEnd

%-----------PROGRAMMING SKILLS-----------
\section{Compétences Techniques}
 \begin{itemize}[leftmargin=0.15in, label={}]
    \small{\item{
     \textbf{Langages}{: Python, Java, C\#, C++, JavaScript, HTML/CSS, SQL (PostgreSQL), MongoDB (NoSQL).} \\
     \textbf{Bibliothèques/Frameworks}{: React, Angular.} \\
     \textbf{Outils de Développement}{: Git, Docker, Azure DevOps, Octopus, Datadog, CI/CD, Linux, Godot, Suite Office.} \\
     \textbf{Montage Vidéo}{: Adobe Premiere Pro, Adobe Photoshop, Pixaki (Pixel Art), Aseprite.} \\
     \textbf{Linguistique}{: Courant en français et en anglais, à l'oral comme à l'écrit.}
    }}
 \end{itemize}

%-------------------------------------------
\end{document}